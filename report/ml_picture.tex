\begin{figure}[ht]
\centering
\begin{tikzpicture}[
  font=\sffamily,
  every matrix/.style={ampersand replacement=\&,column sep=1.5cm,row sep=.5cm},
  source/.style={draw,thick,rounded corners,fill=yellow!20,inner sep=.3cm},
  process2/.style={draw,thick,rounded corners,fill=blue!20,inner sep=.3cm},
  process/.style={draw,thick,circle,fill=orange!20},
  process3/.style={draw,thick,rounded corners,fill=orange!20, inner sep=.3cm},
  sink/.style={source,fill=green!20},
  datastore/.style={draw,very thick,shape=datastore,inner sep=.3cm},
  dots/.style={gray,scale=2},
  to/.style={->,>=stealth',shorten >=1pt,semithick,font=\sffamily\footnotesize},
  dotted/.style={dashed,->,>=stealth',shorten >=1pt,semithick,font=\sffamily\footnotesize},
  every node/.style={align=center}]
% Position the nodes using a matrix layout
\matrix{
  \node[source] (data) {Data};
      \& \node[process] (model) {Model};
        \& \node[sink] (prediction) {Prediction}; \\
};

% Draw the arrows between the nodes and label them.
\draw[to] (data) -- node[midway,above] {Train}
    node[midway,below] {} (model);

\draw[to] (model) -- node[midway,above] {Predict}
    node[midway,below] {} (prediction);

\draw[to] (prediction) to[bend right=50] node[midway,above] {Evaluate}
    node[midway,below] {} (data);
\end{tikzpicture}
\caption[Schematischer Ablauf bei Machine Learning]{Schematischer Ablauf bei Machine Learning}%
\label{fig:ml_process}
\end{figure}