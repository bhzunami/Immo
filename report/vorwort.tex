\section*{Abstract}
Häuserpreise werden heute von Experten geschätzt. Dafür muss ein Experte das Haus begutachten und diverse Kennwerte aufnehmen. Das kostet Geld und ist für beide Parteien mit Aufwand verbunden. Zudem weichen Schätzungen von verschiedenen Makler zum Teil stark voneinander ab.

In diesem Projekt wird analysiert, ob ein Machine Learning Ansatz ähnliche oder sogar bessere Schätzungen abgeben kann. Zu diesem Zweck werden Immobiliendaten direkt aus dem Internet gesammelt und analysiert. Anhand von verschiedenen Machine Learning Algorithmen wird versucht ein exaktes Schätzungsmodell zu berechnen.

% --- Vorwort ----------------------------------------------------- %
\section*{Vorwort}
Das Projekt \emph{Mit Machine Learning Immobilienpreise schätzen} ist eine Forschungsarbeit im Rahmen der Bachelor Arbeit. Es soll untersucht werden, ob öffentliche Immobiliendaten ausreichen um akkurate Schätzungsmodelle zu erstellen.

Die Autoren bedanken sich bei Prof. Dr. Manfred Vogel, er hat das Projekt betreut und unterstützt.

