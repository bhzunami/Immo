\section{Zusammenfassung}
Im Rahmen dieser Arbeit wurde untersucht, wie anhand öffentlichen Immobiliendaten die Kaufpreise von Immobilien mit verschiedenen Machine Leaerning Algorithmen geschätzt werden konnten. Dafür wurden von vier grossen Immobilienportalen 5 Monate lang alle Immobilien, die zum Kauf angeboten wurden, gesammelt. Um nicht von den Anbietern blockiert zu werden, wurde mehrere Proxy Instanzen auf Amazon verwendet. So konnten in dieser Zeit über 120’000 Inserate gesammelt werden.\\
Die Datenqualität der öffentlichen Inseraten waren mässig, durch die Menge konnten trotzdem genügend Inserate mit ausreichender Qualität verwendet werden.\\ 
Zusätzlich zu den Inseraten wurde ortsbezogene Daten vom Bundesamt für Statistik mit den Immobilien verknüpft.
Die gesammelten Daten wurden analysiert und gefiltert. Es zeigte sich, dass viele Inserate nicht verwendet werden konnten, da sie unvollständig und fehlerhaft waren. Somit konnten neben der Beschreibung und den Merkmalen nur vier Kennwerte verwendet werden. Dafür konnten vom Bundesamt für Statistik über 30 Features verwendet werden.\\
Eine ausführliches Feature Engineering ermöglichte es die Performance für die Machine Learning Modelle zu steigern. Dafür wurden Features kombiniert oder transformiert. Schlussendlich standen 2'737 Features zur Verfügung.\\
Eine Analyse der Machine Learning Algorithmen zeigte auf, dass sich lineare Regressions Algorithmen nicht sonderlich gut für unseren Anwendungsfall eignen. Sie schnitten am schlechtesten unter allen Algorithmen ab.\\
Der $K$-Nearest Neighbor hatte eine gute Performance und war auch sehr schnell. Besser waren nur die Baumalgorithmen, insbesondere Extra Trees, XGBoost und AdaBoost. Die besten Ergebnisse bekamen wir durch eine Kombination aller drei Algorithmen. Wobei der AdaBoost die höchste Gewichtung erhielt.\\
Werden die Algorithmn einzeln betrachtet, lieferte AdaBoost die besten Ergebnisse. Neben seiner extrem schnellen Berechnung, lieferte er auch die exaktesten Resultate.

Somit konnte belegt werden, dass die öffentlichen Immobilienplattformen ausreichen um akkurate Schätzungsmodelle zu berechnen. Die Performance konnte mit Hilfe von ortsbezogenen Features und einer guten Outlier Detection gesteigert werden.
