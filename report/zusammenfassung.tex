\section{Zusammenfassung}
% zusammenfassung Kapitel 3, 4, 5
Mittels einer eigens entwickelter Controller App werden die beiden Ansätze für regelbasierte und situationsbasierte Systeme für den Benutzertest zur Verfügung gestellt. Die App integriert sich ins Samsung SmartThings Ökosystem. Mittels einer Rule Engine werden die Regeln und Situationen ausgewertet und sofern nötig Aktionen auf IoT Geräten ausgelöst.\\[2ex]
%
Die Testpersonen wurden sukzessive an jeweils eine von drei vordefinierten Testaufgaben herangeführt und mussten diese lösen. Die Testaufgaben waren aufgeteilt nach einfacher, mittlerer und hoher Komplexität. Alle Aufgaben konnten von allen Testpersonen gelöst werden. Als Merkmal hat sich dabei hervorgehoben, dass der regelbasierte Ansatz bei einfacher Komplexität bevorzugt wurde, während bei der Aufgabe mit hoher Komplexität klar der situationsbasierte Ansatz favorisiert wurde. Die Ergebnisse werden durch Feedback der Testpersonen, sowie durch die statistische Analyse gestärkt.\\[2ex]
%
Der Benutzer sollte daher über das Einsatzgebiet und die Aufgaben des System im Klaren sein, bevor er sich ein IoT System zulegt. Für Lieferanten lohnt es sich, sich in das Thema einzulesen und zumindest experimentell neue Ansätze auszuprobieren. Mit dem Momentum das IoT zurzeit besitzt, eröffnen sich im Markt immer wieder neue Möglichkeiten für die Konfiguration komplexer Systeme.