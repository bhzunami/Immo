%% Machine learning im abschluss
%% Was kann man mit machine learning besser machen
% Keine Unterkapitel
% Was ist unser Beitrag
%  Wie haben wir das Wissen über das iOt System erweitert
% Was empfeheln wir personen die IoT Systeme verwenden
% Was empfeheln wir lieferanten
% Was kann man mit unserer Applikation machen um die beiden Seiten zusammen zu bringen
% Wie soll man grundsätzlich Konfigurationen von IoT Verbessern
% -> Testumgebungen Live feedback
% Perspektiven
\section{Diskussion}
Es hat sich herausgestellt, dass der situationsbasierte Ansatz bei komplexeren Systemen einen klaren Vorteil bietet. Ob sich der situationsbasierte Ansatz wirklich lohnt muss von System zu System separat beurteilt werden. Der Benutzer sollte sich daher über das Einsatzgebiet, beziehungsweise die Aufgaben des Systems im Klaren sein, bevor er sich ein IoT System zulegt. Auch sollte er sich informieren welche Schnittstellen ein solches IoT System anbietet, damit er weiss welche Applikationen, wie zum Beispiel IFTTT, er verwenden kann.\\
Für einen Lieferanten kann es sinnvoll sein, eine situationsbasierte Konfiguration anzubieten. Sie erlauben dem Benutzer mehr Freiheiten. Jedoch erhöht sich somit auch die Komplexität für das Einrichten eines IoT Systems. Wichtig für ein IoT System ist die Transparenz. Wenn der Benutzer nachvollziehen kann wie das System funktioniert, wird auch das Vertrauen ins System gestärkt. Grundsätzlich sollte ein IoT System so einfach wie möglich für den Benutzer sein. Aber trotzdem flexibel genug um den Benutzer nicht einzuschränken. Je nach Anwendungsfall macht es Sinn den regelbasierten Ansatz als Standard festzulegen und zusätzlich eine Expertensicht mit dem situationsbasierten Ansatz anzubieten. Wird dies nicht von einem Lieferanten unterstützt, können Mobile- sowie auch Webapplikationen, wie in dieser Arbeit gezeigt, verwendet werden um zu dieser Unterstützung zu gelangen. Die Umsetzung hängt stark von der Schnittstelle zum Lieferanten ab. Bei Samsung SmartThings ist diese gut.\\[2ex]
%
% Hätte man das Resultat schon vorher wissen können
% Satz über Komplex
% Umgehen konnten
% Bedeutung
Dass der situationsbasierte Ansatz bei komplexen Systemen bevorzugt wird, konnte nur bedingt vorhergesagt werden. Es war klar, dass komplexe Systeme mit einem regelbasierten Ansatz gar nicht umsetzbar waren. Somit hatte hier der situationsbasierte Ansatz einen Vorteil. Nicht klar war jedoch, ob die Benutzer mit dem situationsbasierten Ansatz etwas anfangen konnten. Schnell hatte sich aber gezeigt, dass die Testpersonen diesen Ansatz ohne Probleme umsetzen konnten. Obwohl der situationsbasierte Ansatz bei einem einfachen System einen kleinen Mehraufwand mit sich brachte, reduzierte er die Komplexität eines Systems. Durch diese Reduktion unterstützt der Ansatz den Benutzer bei der Umsetzung und gibt ihm eine gute Übersicht über sein System.\\
Zudem wurde eine Komplexitäts-Definition aufgestellt, um Systeme einzuordnen. Diese Kategorisierung hilft dem Benutzer den geeigneten Ansatz zu bestimmen.\\[2ex]
%
Da gewisse Testpersonen bei dem situationsbasierten Ansatz Probleme mit den Übergängen hatten, wäre es eventuell sinnvoll, diese Übergänge nicht vom Benutzer selber bestimmen zu lassen, sondern mittels Machine Learning  eigenständig zu erkennen.\\
So müsste der Benutzer nur seine Situationen und Regeln definieren. Das System in einen Learning-Modus versetzt um die Übergänge zu lernen. Nach einer gewissen Zeit kann das System entscheiden, ob es einen Situationswechsel gibt oder nicht.\\
Es lohnt sich sicher noch eine repräsentative Studie mit mehr Testpersonen durchzuführen, um ein qualitativ besseres Resultat zu erhalten. Es rentiert sich auf jeden Fall den situationsbasierten Ansatz als Prototyp weiter zu entwickeln und weitere Erkenntnisse zu sammeln. Somit kann es gut sein, dass man in Zukunft von Situationen mit Regeln spricht, anstatt nur von Regeln.