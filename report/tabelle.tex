\begin{center}
  \begin{table}[ht]
    \begin{tabular}{p{3cm} | p{6cm} | p{6cm}}  
      \textbf{Situation} & \textbf{If this} & \textbf{Then that} \\
      \hline
      Ich bin zuhause & 
           Wenn ich das Haus verlasse und das Licht beim Eingang ist an & 
           dann schalte das Licht aus\\ \cline{2-3}
      & Wenn ich ins Wohnzimmer komme und das Licht ist ab & 
           dann schalte das Licht im Wohnzimmer an\\
      \hline
      Ich bin nicht zuhause  & 
           Wenn ich nach Hause komme und das Licht im Eingang nicht an ist &
           dann schalte das Licht ein\\ \cline{2-3}
      & Wenn keine Bewegung zu Hause ist & 
           dann schalte alle Lichter aus\\
      \hline
    \end{tabular}
    \caption{\label{tab:situation_ifttt}Situation mit IFTTT.}
  \end{table}
\end{center}



Für die Umsetzung des Szenarios stehen dem Benutzer diverse Sensoren, wie zum Beispiel einen Bewegungssensor oder ein Kontaktsensor, zur Verfügung.\\
Um sein Eigenheim zu Überwachen, kann der Benutzer die vorinstallierten \textit{Monitor Regeln} von Samsung SmartThings verwenden. Diese Regeln Überwachen die Sensoren auf Änderungen und lösen eine oder mehrere vordefinierten Aktionen, wie zum Beispiel eine Push Benachrichtigung, aus.\\
Neben den vorinstallierten Regeln kann der Benutzer den Marktplace von \textit{Samsung SmartThings} verwenden. Dieser ist jedoch noch sehr klein und hat etwa 15 Applikation \footnote{Stand November 2016} im Bereich \textit{Home Security / Monitoring}.\\
Hauptsächlich finden sich in diesem Bereich einfache Benachrichtigungs-, Wetter- oder Verriegelungs Applikationen.\\
Bei beiden Varianten kann man nur erschwert oder gar nicht ein komplexeres System abbilden, das mehreren Sensoren beinhaltet die abhängig voneinander sind.\\[2ex]
%
Zudem kommt, dass die Samsung SmartThings Mobile Applikation die einzige Schnittstelle zu den Sensoren ist.
Hat der Benutzer mehr als 5 Sensoren im Betrieb, kann er schnell die Übersicht verlieren. Infolgedessen kann der Benutzer nur umständlich ermitteln, wo ein bestimmter Sensor verwendet wird.\\[2ex]


Um eine bessere Vorstellung zu erhalten, wollen wir drei Beispiele aufzeigen.\\[2ex]
% Grundidee eines was Komplex ist nur in einem Satz beschreiben
\textbf{Einfache Komplexität}:\\
Eine Person hat in einem Raum einen Bewegungssensor und einen Schalter. Jedesmal wenn er den Raum betritt, wird der Bewegungssensor aktiv und das Licht geht an. Ist der Bewegungssensor inaktiv, wird das Licht wieder ausgeschaltet.\\
Die Regel, Licht geht an wenn Bewegung, hat keine Abhängigkeit zu anderen Regeln oder Sensoren und hat deshalb eine einfache Komplexität.\\[2ex]
\textbf{Mittlere Komplexität}:\\
Eine Person hat in zwei Räumen 5 verschiedene Sensoren. Dazu kommt, dass die Sensoren je nach Situation ein anderes Verhalten haben. So will die Person, wenn sie nicht zuhause ist, benachrichtigt werden wenn sich ein Sensor ändert. Ist sie aber zu hause, sollte keine Benachrichtigung stattfinden.\\
Die Komplexität wird hier erhöht, indem dasselbe Event auf einem Sensor, situationsabhängige Aktionen ausführt.\\[2ex]
\textbf{Hohe Komplexität}:\\
Wir haben ein Haus mit einer Kleinfamilie. Sie möchten das Haus mit diversen Sensoren überwachen.\\
Wenn niemand Zuhause ist, wird das Schloss an der Türe verriegelt. Wird die Türe aufgebrochen werden die Eltern benachrichtigt. Die Kinder haben noch keinen Schlüssel, können aber über die Gartentüre ins Haus.\\
In der Nacht sollte das Haus im Erdgeschoss überwacht und falls das Licht noch brennt, ausgeschaltet werden.\\
Wir sehen, dass in diesem Beispiel diverse Situationen entstehen bei denen diverse Aktionen ausgeführt werden. Es können ganze Abläufe entstehen, die wir im Kapitel 4 näher erläutern. Für uns ist das ein IoT-System mit hoher Komplexität.\\[2ex]